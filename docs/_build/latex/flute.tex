%% Generated by Sphinx.
\def\sphinxdocclass{report}
\documentclass[a4paper,10pt,english]{sphinxmanual}
\ifdefined\pdfpxdimen
   \let\sphinxpxdimen\pdfpxdimen\else\newdimen\sphinxpxdimen
\fi \sphinxpxdimen=.75bp\relax
\ifdefined\pdfimageresolution
    \pdfimageresolution= \numexpr \dimexpr1in\relax/\sphinxpxdimen\relax
\fi
%% let collapsable pdf bookmarks panel have high depth per default
\PassOptionsToPackage{bookmarksdepth=5}{hyperref}

\PassOptionsToPackage{warn}{textcomp}
\usepackage[utf8]{inputenc}
\ifdefined\DeclareUnicodeCharacter
% support both utf8 and utf8x syntaxes
  \ifdefined\DeclareUnicodeCharacterAsOptional
    \def\sphinxDUC#1{\DeclareUnicodeCharacter{"#1}}
  \else
    \let\sphinxDUC\DeclareUnicodeCharacter
  \fi
  \sphinxDUC{00A0}{\nobreakspace}
  \sphinxDUC{2500}{\sphinxunichar{2500}}
  \sphinxDUC{2502}{\sphinxunichar{2502}}
  \sphinxDUC{2514}{\sphinxunichar{2514}}
  \sphinxDUC{251C}{\sphinxunichar{251C}}
  \sphinxDUC{2572}{\textbackslash}
\fi
\usepackage{cmap}
\usepackage[T1]{fontenc}
\usepackage{amsmath,amssymb,amstext}
\usepackage{babel}



\usepackage{tgtermes}
\usepackage{tgheros}
\renewcommand{\ttdefault}{txtt}



\usepackage[Bjarne]{fncychap}
\usepackage{sphinx}

\fvset{fontsize=auto}
\usepackage{geometry}


% Include hyperref last.
\usepackage{hyperref}
% Fix anchor placement for figures with captions.
\usepackage{hypcap}% it must be loaded after hyperref.
% Set up styles of URL: it should be placed after hyperref.
\urlstyle{same}

\addto\captionsenglish{\renewcommand{\contentsname}{Contents:}}

\usepackage{sphinxmessages}
\setcounter{tocdepth}{1}



\title{FLUTE}
\date{May 21, 2021}
\release{version 1.0}
\author{Emilee Holtzapple}
\newcommand{\sphinxlogo}{\vbox{}}
\renewcommand{\releasename}{Release}
\makeindex
\begin{document}

\pagestyle{empty}
\sphinxmaketitle
\pagestyle{plain}
\sphinxtableofcontents
\pagestyle{normal}
\phantomsection\label{\detokenize{index::doc}}


\sphinxAtStartPar
Understanding disease at the cellular level requires detailed knowledge of signaling networks. To aid in this task, many advances have been made in the field of natural language processing (NLP) to extract signaling events from biomedical literature.

\sphinxAtStartPar
However, even state\sphinxhyphen{}of\sphinxhyphen{}the\sphinxhyphen{}art NLP methods incorrectly interpret some signaling events described in the literature.

\noindent\sphinxincludegraphics[width=400\sphinxpxdimen]{{figure_1}.png}

\sphinxAtStartPar
The FiLter for Understanding True Events (FLUTE) tool seeks to identify high\sphinxhyphen{}confidence signaling events from biomedical NLP output by comparing with existing biological databases. As such, FLUTE can reliably determine the confidence in the biomolecular events extracted by NLP methods and at the same time provide a speedup in event filtering by three orders of magnitude.


\chapter{Installation instructions}
\label{\detokenize{Installation:installation-instructions}}\label{\detokenize{Installation::doc}}

\section{MySQL}
\label{\detokenize{Installation:mysql}}\begin{enumerate}
\sphinxsetlistlabels{\arabic}{enumi}{enumii}{}{.}%
\item {} 
\sphinxAtStartPar
Download the appropriate distribution of \sphinxhref{https://dev.mysql.com/doc/mysql-installation-excerpt/5.7/en/}{MySQL}.

\item {} 
\sphinxAtStartPar
Restart your computer and add to path if necessary.

\item {} 
\sphinxAtStartPar
From the command line, access the MySQL environment by typing:

\begin{sphinxVerbatim}[commandchars=\\\{\}]
\PYG{n}{mysql} \PYG{o}{\PYGZhy{}}\PYG{n}{u} \PYG{n}{root}
\end{sphinxVerbatim}

\end{enumerate}

\sphinxAtStartPar
If the first prompt fails, you may need to enter the password associated with your computer user account:

\begin{sphinxVerbatim}[commandchars=\\\{\}]
\PYG{n}{mysql} \PYG{o}{\PYGZhy{}}\PYG{n}{u} \PYG{n}{root} \PYG{o}{\PYGZhy{}}\PYG{n}{p}
\end{sphinxVerbatim}
\begin{enumerate}
\sphinxsetlistlabels{\arabic}{enumi}{enumii}{}{.}%
\setcounter{enumi}{2}
\item {} 
\sphinxAtStartPar
You may choose to create a local username and password to keep your database private.

\item {} 
\sphinxAtStartPar
Install \sphinxhref{https://dev.mysql.com/doc/connector-python/en/}{MySQL Python connector}.

\end{enumerate}


\section{FLUTE database}
\label{\detokenize{Installation:flute-database}}\begin{enumerate}
\sphinxsetlistlabels{\arabic}{enumi}{enumii}{}{.}%
\item {} 
\sphinxAtStartPar
Un\sphinxhyphen{}zip the FLUTE.sql file downloaded from BitBucket.

\item {} 
\sphinxAtStartPar
Log in to the MySQL environment using your username and password.

\item {} 
\sphinxAtStartPar
From there, create an empty database.

\item {} 
\sphinxAtStartPar
Log back out, and again from the command line:

\begin{sphinxVerbatim}[commandchars=\\\{\}]
\PYG{n}{mysql} \PYG{o}{\PYGZhy{}}\PYG{n}{u} \PYG{n}{username} \PYG{o}{\PYGZhy{}}\PYG{n}{p} \PYG{n}{database\PYGZus{}name} \PYG{o}{\PYGZlt{}} \PYG{n}{FLUTE}\PYG{o}{.}\PYG{n}{sql}
\end{sphinxVerbatim}

\item {} 
\sphinxAtStartPar
If you created a username and password, this will be your username in the above command, but do not enter your password above! Once you hit enter, it will prompt you for the password.

\item {} 
\sphinxAtStartPar
You can now run the “run\_FLUTE.py” script, you will need to enter the database, host, username, etc. as an argument from the command line.

\end{enumerate}


\chapter{FLUTE usage}
\label{\detokenize{Usage:flute-usage}}\label{\detokenize{Usage::doc}}\begin{enumerate}
\sphinxsetlistlabels{\arabic}{enumi}{enumii}{}{.}%
\item {} 
\sphinxAtStartPar
To filter interactions, run “run\_FLUTE.py”. You must have Python3 installed.

\item {} \begin{description}
\item[{The script takes several parameters:}] \leavevmode\begin{enumerate}
\sphinxsetlistlabels{\Alph}{enumii}{enumiii}{}{.}%
\item {} 
\sphinxAtStartPar
MySQL username

\item {} 
\sphinxAtStartPar
MySQL password

\item {} 
\sphinxAtStartPar
Host name \sphinxhyphen{} “localhost” for MacOSX, desktop name for Windows

\item {} 
\sphinxAtStartPar
Database name (see step 3 from FLUTE DB installation instructions)

\item {} 
\sphinxAtStartPar
Input filename

\item {} 
\sphinxAtStartPar
Output filename for interactions

\item {} 
\sphinxAtStartPar
Output filename for scores

\end{enumerate}

\end{description}

\item {} 
\sphinxAtStartPar
Input files must have the following headers:


\begin{savenotes}\sphinxattablestart
\centering
\begin{tabulary}{\linewidth}[t]{|T|T|T|T|T|T|T|}
\hline

\sphinxAtStartPar
RegulatedName
&
\sphinxAtStartPar
RegulatedID
&
\sphinxAtStartPar
RegulatedType
&
\sphinxAtStartPar
RegulatorName
&
\sphinxAtStartPar
RegulatorID
&
\sphinxAtStartPar
RegulatorType
&
\sphinxAtStartPar
PaperID
\\
\hline
\end{tabulary}
\par
\sphinxattableend\end{savenotes}

\item {} 
\sphinxAtStartPar
Output files include list of reading interactions that pass filtration, and the filtration scores for those filtered interactions.

\end{enumerate}


\chapter{run\_FLUTE}
\label{\detokenize{run_FLUTE:run-flute}}\label{\detokenize{run_FLUTE::doc}}
\sphinxAtStartPar
This page describes the script that accesses the FLUTE database.
The functions in this module ground element names and check against the FLUTE database.


\section{Functions}
\label{\detokenize{run_FLUTE:functions}}\index{getRelatedPapers() (in module run\_FLUTE)@\spxentry{getRelatedPapers()}\spxextra{in module run\_FLUTE}}

\begin{fulllineitems}
\phantomsection\label{\detokenize{run_FLUTE:run_FLUTE.getRelatedPapers}}\pysiglinewithargsret{\sphinxcode{\sphinxupquote{run\_FLUTE.}}\sphinxbfcode{\sphinxupquote{getRelatedPapers}}}{\emph{\DUrole{n}{db\_user}}, \emph{\DUrole{n}{db\_pass}}, \emph{\DUrole{n}{db\_host}}, \emph{\DUrole{n}{db\_name}}, \emph{\DUrole{n}{prot}}}{}
\sphinxAtStartPar
This function retrieves related papers based on a protein name.
\begin{description}
\item[{db\_user: str}] \leavevmode
\sphinxAtStartPar
Name of the MySQL user where the FLUTE DB is stored.

\item[{db\_pass: str}] \leavevmode
\sphinxAtStartPar
Password for the MySQL user where the FLUTE DB is stored.

\item[{db\_host: str }] \leavevmode
\sphinxAtStartPar
Host name for the local machine where the coopy of the FLUTE DB is stored.

\item[{db\_name: str}] \leavevmode
\sphinxAtStartPar
Name of the local copy of the FLUTE DB.

\item[{prot: str}] \leavevmode
\sphinxAtStartPar
Input protein

\end{description}

\sphinxAtStartPar
Saves a file of related paper IDs

\end{fulllineitems}

\index{getRelatedInts() (in module run\_FLUTE)@\spxentry{getRelatedInts()}\spxextra{in module run\_FLUTE}}

\begin{fulllineitems}
\phantomsection\label{\detokenize{run_FLUTE:run_FLUTE.getRelatedInts}}\pysiglinewithargsret{\sphinxcode{\sphinxupquote{run\_FLUTE.}}\sphinxbfcode{\sphinxupquote{getRelatedInts}}}{\emph{\DUrole{n}{db\_user}}, \emph{\DUrole{n}{db\_pass}}, \emph{\DUrole{n}{db\_host}}, \emph{\DUrole{n}{db\_name}}, \emph{\DUrole{n}{f}}}{}
\sphinxAtStartPar
This function retrieves interactions from the same papers as
\begin{description}
\item[{db\_user: str}] \leavevmode
\sphinxAtStartPar
Name of the MySQL user where the FLUTE DB is stored.

\item[{db\_pass: str}] \leavevmode
\sphinxAtStartPar
Password for the MySQL user where the FLUTE DB is stored.

\item[{db\_host: str }] \leavevmode
\sphinxAtStartPar
Host name for the local machine where the coopy of the FLUTE DB is stored.

\item[{db\_name: str}] \leavevmode
\sphinxAtStartPar
Name of the local copy of the FLUTE DB.

\item[{f: str}] \leavevmode
\sphinxAtStartPar
File name of the list of papers

\end{description}

\end{fulllineitems}

\index{getRecentPapers() (in module run\_FLUTE)@\spxentry{getRecentPapers()}\spxextra{in module run\_FLUTE}}

\begin{fulllineitems}
\phantomsection\label{\detokenize{run_FLUTE:run_FLUTE.getRecentPapers}}\pysiglinewithargsret{\sphinxcode{\sphinxupquote{run\_FLUTE.}}\sphinxbfcode{\sphinxupquote{getRecentPapers}}}{\emph{\DUrole{n}{f}}}{}
\sphinxAtStartPar
This function should search the OA file and find all interactions occuring in papers less than X years old
\begin{description}
\item[{f: str}] \leavevmode
\sphinxAtStartPar
Input filename that contains list of interactions

\end{description}

\sphinxAtStartPar
None

\end{fulllineitems}

\index{getDups() (in module run\_FLUTE)@\spxentry{getDups()}\spxextra{in module run\_FLUTE}}

\begin{fulllineitems}
\phantomsection\label{\detokenize{run_FLUTE:run_FLUTE.getDups}}\pysiglinewithargsret{\sphinxcode{\sphinxupquote{run\_FLUTE.}}\sphinxbfcode{\sphinxupquote{getDups}}}{\emph{\DUrole{n}{f}}}{}
\sphinxAtStartPar
This function calculates the number of occurences of an interaction in a reading set.
\begin{description}
\item[{f: str}] \leavevmode
\sphinxAtStartPar
Filename of the list of interactions to be counted.

\end{description}

\sphinxAtStartPar
None

\end{fulllineitems}

\index{getArgs() (in module run\_FLUTE)@\spxentry{getArgs()}\spxextra{in module run\_FLUTE}}

\begin{fulllineitems}
\phantomsection\label{\detokenize{run_FLUTE:run_FLUTE.getArgs}}\pysiglinewithargsret{\sphinxcode{\sphinxupquote{run\_FLUTE.}}\sphinxbfcode{\sphinxupquote{getArgs}}}{}{}
\end{fulllineitems}

\index{convID() (in module run\_FLUTE)@\spxentry{convID()}\spxextra{in module run\_FLUTE}}

\begin{fulllineitems}
\phantomsection\label{\detokenize{run_FLUTE:run_FLUTE.convID}}\pysiglinewithargsret{\sphinxcode{\sphinxupquote{run\_FLUTE.}}\sphinxbfcode{\sphinxupquote{convID}}}{\emph{\DUrole{n}{db\_user}}, \emph{\DUrole{n}{db\_pass}}, \emph{\DUrole{n}{db\_host}}, \emph{\DUrole{n}{db\_name}}, \emph{\DUrole{n}{X}}}{}
\sphinxAtStartPar
This function uses the FLUTE DB to ground interactions
\begin{description}
\item[{db\_user: str}] \leavevmode
\sphinxAtStartPar
Name of the MySQL user where the FLUTE DB is stored.

\item[{db\_pass: str}] \leavevmode
\sphinxAtStartPar
Password for the MySQL user where the FLUTE DB is stored.

\item[{db\_host: str }] \leavevmode
\sphinxAtStartPar
Host name for the local machine where the coopy of the FLUTE DB is stored.

\item[{db\_name: str}] \leavevmode
\sphinxAtStartPar
Name of the local copy of the FLUTE DB.

\item[{X: numpy array}] \leavevmode
\sphinxAtStartPar
Array containing all grounded interactions

\end{description}
\begin{description}
\item[{X: numpy array}] \leavevmode
\sphinxAtStartPar
Grounded ineractions

\end{description}

\end{fulllineitems}

\index{findInts() (in module run\_FLUTE)@\spxentry{findInts()}\spxextra{in module run\_FLUTE}}

\begin{fulllineitems}
\phantomsection\label{\detokenize{run_FLUTE:run_FLUTE.findInts}}\pysiglinewithargsret{\sphinxcode{\sphinxupquote{run\_FLUTE.}}\sphinxbfcode{\sphinxupquote{findInts}}}{\emph{\DUrole{n}{db\_user}}, \emph{\DUrole{n}{db\_pass}}, \emph{\DUrole{n}{db\_host}}, \emph{\DUrole{n}{db\_name}}, \emph{\DUrole{n}{ints}}, \emph{\DUrole{n}{es}}, \emph{\DUrole{n}{ts}}, \emph{\DUrole{n}{ds}}}{}
\sphinxAtStartPar
This function uses the FLUTE DB to filter interactions
\begin{description}
\item[{db\_user: str}] \leavevmode
\sphinxAtStartPar
Name of the MySQL user where the FLUTE DB is stored.

\item[{db\_pass: str}] \leavevmode
\sphinxAtStartPar
Password for the MySQL user where the FLUTE DB is stored.

\item[{db\_host: str }] \leavevmode
\sphinxAtStartPar
Host name for the local machine where the coopy of the FLUTE DB is stored.

\item[{db\_name: str}] \leavevmode
\sphinxAtStartPar
Name of the local copy of the FLUTE DB.

\item[{X: numpy array}] \leavevmode
\sphinxAtStartPar
Array containing all filtered interactions

\end{description}
\begin{description}
\item[{X: numpy array}] \leavevmode
\sphinxAtStartPar
Filtered interactions

\end{description}

\end{fulllineitems}

\index{uniOnly() (in module run\_FLUTE)@\spxentry{uniOnly()}\spxextra{in module run\_FLUTE}}

\begin{fulllineitems}
\phantomsection\label{\detokenize{run_FLUTE:run_FLUTE.uniOnly}}\pysiglinewithargsret{\sphinxcode{\sphinxupquote{run\_FLUTE.}}\sphinxbfcode{\sphinxupquote{uniOnly}}}{\emph{\DUrole{n}{allInts}}}{}
\sphinxAtStartPar
This function returns only proteins
\begin{description}
\item[{allInts: numpy array}] \leavevmode
\sphinxAtStartPar
All interactions from the input file

\end{description}
\begin{description}
\item[{rel\_ints: numpy array}] \leavevmode
\sphinxAtStartPar
Protein\sphinxhyphen{}protein interactions only

\end{description}

\end{fulllineitems}

\index{getChem() (in module run\_FLUTE)@\spxentry{getChem()}\spxextra{in module run\_FLUTE}}

\begin{fulllineitems}
\phantomsection\label{\detokenize{run_FLUTE:run_FLUTE.getChem}}\pysiglinewithargsret{\sphinxcode{\sphinxupquote{run\_FLUTE.}}\sphinxbfcode{\sphinxupquote{getChem}}}{\emph{\DUrole{n}{X}}}{}
\sphinxAtStartPar
This function returns only protein\sphinxhyphen{}chemical interactions
\begin{description}
\item[{X: numpy array}] \leavevmode
\sphinxAtStartPar
All interactions from the input file

\end{description}
\begin{description}
\item[{X: numpy array}] \leavevmode
\sphinxAtStartPar
Protein\sphinxhyphen{}chemical interactions only

\end{description}

\end{fulllineitems}

\index{getGo() (in module run\_FLUTE)@\spxentry{getGo()}\spxextra{in module run\_FLUTE}}

\begin{fulllineitems}
\phantomsection\label{\detokenize{run_FLUTE:run_FLUTE.getGo}}\pysiglinewithargsret{\sphinxcode{\sphinxupquote{run\_FLUTE.}}\sphinxbfcode{\sphinxupquote{getGo}}}{\emph{\DUrole{n}{a}}}{}
\sphinxAtStartPar
This function returns only protein\sphinxhyphen{}biological process interactions
\begin{description}
\item[{a: numpy array}] \leavevmode
\sphinxAtStartPar
All interactions from the input file

\end{description}
\begin{description}
\item[{a: numpy array}] \leavevmode
\sphinxAtStartPar
Protein\sphinxhyphen{}biological process interactions only

\end{description}

\end{fulllineitems}



\section{Dependencies}
\label{\detokenize{run_FLUTE:dependencies}}
\sphinxAtStartPar
\sphinxstylestrong{Python}:

\sphinxAtStartPar
\sphinxhref{https://pandas.pydata.org}{pandas} library

\sphinxAtStartPar
\sphinxhref{https://github.com/python/cpython/blob/3.9/Lib/csv.py}{csv} module

\sphinxAtStartPar
\sphinxhref{https://numpy.org}{numpy} library

\sphinxAtStartPar
\sphinxhref{https://dev.mysql.com/doc/connector-python/en/}{MySQL Connector for Python3} library

\sphinxAtStartPar
\sphinxhref{https://docs.python.org/3/library/argparse.html}{argparse} library
\sphinxhref{https://docs.python.org/3/library/re.html}{re} library


\chapter{Legal}
\label{\detokenize{Legal:legal}}\label{\detokenize{Legal::doc}}
\sphinxAtStartPar
Add here any information concerning usage, dowloads, and repurposing.


\chapter{Licensing and funding}
\label{\detokenize{Funding:licensing-and-funding}}\label{\detokenize{Funding::doc}}
\sphinxAtStartPar
Supported by DARPA award..


\chapter{Indices and tables}
\label{\detokenize{index:indices-and-tables}}\begin{itemize}
\item {} 
\sphinxAtStartPar
\DUrole{xref,std,std-ref}{genindex}

\item {} 
\sphinxAtStartPar
\DUrole{xref,std,std-ref}{modindex}

\item {} 
\sphinxAtStartPar
\DUrole{xref,std,std-ref}{search}

\end{itemize}



\renewcommand{\indexname}{Index}
\printindex
\end{document}